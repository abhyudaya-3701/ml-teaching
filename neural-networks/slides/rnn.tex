\documentclass[usenames,dvipsnames]{beamer}
\usepackage{../../shared/styles/custom}
\usepackage{../../shared/styles/conventions}

\title{Recurrent Neural Networks}
\date{\today}
\author{Nipun Batra and teaching staff}
\institute{IIT Gandhinagar}

% Pop quiz counter is defined in theme-nipun.sty
\setcounter{popquiz}{0}

\begin{document}
	\maketitle
	
	% Table of Contents
	\begin{frame}{Table of Contents}
		\tableofcontents[hideallsubsections]
	\end{frame}
	
	\section{Introduction and Motivation}
	
	\begin{frame}{Why Sequential Data Matters}
		\begin{examplebox}{Sequential Data Examples}
			\begin{itemize}
				\item \textbf{Text}: "The quick brown fox jumps..."
				\item \textbf{Speech}: Audio waveforms over time
				\item \textbf{Stock Prices}: Daily market values
				\item \textbf{Weather}: Temperature, humidity over days
			\end{itemize}
		\end{examplebox}
		
		\begin{alertbox}{Challenge}
			Traditional feedforward networks treat inputs independently - they can't capture \textbf{temporal dependencies}.
		\end{alertbox}
	\end{frame}
	
	\section{Basic RNN Architecture}
	
	\begin{frame}{Simple RNN Cell}
		\begin{definitionbox}{RNN Equations}
			\begin{align}
				h_t &= \tanh(W_{hh} h_{t-1} + W_{xh} x_t + b_h) \\
				y_t &= W_{hy} h_t + b_y
			\end{align}
		\end{definitionbox}
		
		\begin{keypointsbox}{Important}
			\begin{itemize}
				\item Same weights shared across all time steps
				\item Hidden state acts as ``memory''
				\item Can process variable length sequences
			\end{itemize}
		\end{keypointsbox}
	\end{frame}
	
	\popquiz{
			\textbf{What happens to gradients in simple RNNs during backpropagation?}
			\begin{enumerate}[A)]
				\item They remain constant
				\item They can explode or vanish
				\item They always improve
				\item They disappear completely
			\end{enumerate}
	}{{\color{magenta}\textbf{B) They can explode or vanish}}}

	\begin{frame}{The Issue}
	\begin{alertbox}{The Gradient Problem}
			\begin{itemize}
				\item Gradients multiply by $W_{hh}$ at each time step
				\item If $||W_{hh}|| > 1$: Exploding gradients
				\item If $||W_{hh}|| < 1$: Vanishing gradients
			\end{itemize}
		\end{alertbox}
	\end{frame}
	\section{RNN Applications}
	
	\begin{frame}{Sentiment Analysis (Many-to-One)}
		\begin{examplebox}{Sequence Classification}
			\begin{itemize}
				\item Input: "This movie is great!"
				\item Process each word sequentially
				\item Output: Positive/Negative sentiment
			\end{itemize}
		\end{examplebox}
		
		\begin{keypointsbox}{Applications}
			Document classification, spam detection, review analysis
		\end{keypointsbox}
	\end{frame}
	
	\begin{frame}{Machine Translation (Many-to-Many)}
		\begin{examplebox}{Sequence-to-Sequence}
			\begin{itemize}
				\item \textbf{Encoder}: French → "Je suis étudiant"
				\item \textbf{Context}: Hidden representation
				\item \textbf{Decoder}: English → "I am student"
			\end{itemize}
		\end{examplebox}
	\end{frame}
	
	\section{Advanced Variants}
	
	\begin{frame}{LSTM: Long Short-Term Memory}
		\begin{definitionbox}{LSTM Key Idea}
			Use \textbf{gates} to control information flow:
			\begin{itemize}
				\item \textbf{Forget gate}: What to remove from memory
				\item \textbf{Input gate}: What new information to store  
				\item \textbf{Output gate}: What parts of memory to output
			\end{itemize}
		\end{definitionbox}
		
		\begin{theorembox}{Advantage}
			LSTM gates solve the vanishing gradient problem by allowing gradients to flow unchanged through time.
		\end{theorembox}
	\end{frame}
	
	\begin{frame}{GRU: Gated Recurrent Unit}
		\begin{keypointsbox}{GRU vs LSTM}
			\begin{itemize}
				\item Simpler: Only 2 gates instead of 3
				\item Faster training and inference
				\item Often performs similarly to LSTM
				\item Good starting point for many applications
			\end{itemize}
		\end{keypointsbox}
	\end{frame}
	
	\section{Modern Context}
	
	\begin{frame}{From RNNs to Transformers}
		\begin{theorembox}{Why Transformers Won}
			\begin{itemize}
				\item \textbf{Parallelizable}: No sequential dependency
				\item \textbf{Long-range dependencies}: Attention mechanism
				\item \textbf{Scalable}: Works well with large datasets
				\item \textbf{Transfer learning}: Pre-trained models (GPT, BERT)
			\end{itemize}
		\end{theorembox}
	\end{frame}
	
	\begin{frame}{When to Still Use RNNs}
		\footnotesize
		\begin{definitionbox}{RNN Strengths}
			\begin{itemize}
				\item \textbf{Memory efficiency}: Constant memory usage
				\item \textbf{Online processing}: Can process streaming data
				\item \textbf{Small datasets}: Less prone to overfitting
				\item \textbf{Simple problems}: Often sufficient
			\end{itemize}
		\end{definitionbox}
		
		\begin{examplebox}{Modern Applications}
			\begin{itemize}
				\item Real-time speech recognition
				\item IoT sensor data processing  
				\item Mobile applications
				\item Control systems
			\end{itemize}
		\end{examplebox}
	\end{frame}
	
	\section{Summary}
	
	\begin{frame}{Key Takeaways}
		\begin{keypointsbox}{What we learned}
			\begin{enumerate}
				\item RNNs process sequential data with memory
				\item Simple RNNs suffer from gradient problems
				\item LSTM and GRU solve long-term dependencies
				\item Training uses Backpropagation Through Time
				\item Transformers have largely replaced RNNs for NLP
			\end{enumerate}
		\end{keypointsbox}
		
		\begin{theorembox}{The Big Picture}
			RNNs introduced \textbf{sequential processing with memory} to deep learning, paving the way for modern language models.
		\end{theorembox}
	\end{frame}

\end{document}