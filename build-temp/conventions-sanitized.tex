% =============================================================================
% ML TEACHING MATHEMATICAL NOTATION CONVENTIONS
% =============================================================================
% Based on standard ML textbooks: Murphy's "Machine Learning: A Probabilistic Perspective",
% Bishop's "Pattern Recognition and Machine Learning", and "Mathematics for Machine Learning"

% =============================================================================
% CORE NOTATION STANDARDS
% =============================================================================

% SCALARS: Regular italics (lowercase for variables, uppercase for constants)
% Examples: x, y, n, d, k, \theta, \alpha, \lambda, \sigma

% VECTORS: Bold lowercase letters
% Examples: \mathbf{x}, \mathbf{w}, \mathbf{\mu}, \mathbf{\theta}

% MATRICES: Bold uppercase letters
% Examples: \mathbf{X}, \mathbf{W}, \mathbf{\Sigma}, \mathbf{\Lambda}

% SETS: Calligraphic uppercase
% Examples: \mathcal{D}, \mathcal{X}, \mathcal{Y}

% SPACES: Blackboard bold
% Examples: \mathbb{R}, \mathbb{Z}, \mathbb{N}

% =============================================================================
% VECTOR NOTATION (bold lowercase)
% =============================================================================

\providecommand{\vx}{\mathbf{x}}        % Input vector
\providecommand{\vy}{\mathbf{y}}        % Output vector
\providecommand{\vw}{\mathbf{w}}        % Weight vector
\providecommand{\vb}{\mathbf{b}}        % Bias vector
\providecommand{\vh}{\mathbf{h}}        % Hidden vector
\providecommand{\vz}{\mathbf{z}}        % Latent vector
\providecommand{\vf}{\mathbf{f}}        % Function vector
\providecommand{\vg}{\mathbf{g}}        % Gradient vector
\providecommand{\vu}{\mathbf{u}}        % Generic vector u
\providecommand{\vv}{\mathbf{v}}        % Generic vector v
\providecommand{\vzero}{\mathbf{0}}     % Zero vector
\providecommand{\vone}{\mathbf{1}}      % Ones vector

% Greek vectors (bold)
\providecommand{\vmu}{\boldsymbol{\mu}}     % Mean vector
\providecommand{\vtheta}{\boldsymbol{\theta}} % Parameter vector
\providecommand{\vlambda}{\boldsymbol{\lambda}} % Lambda vector
\providecommand{\valpha}{\boldsymbol{\alpha}}   % Alpha vector
\providecommand{\vbeta}{\boldsymbol{\beta}}     % Beta vector
\providecommand{\vxi}{\boldsymbol{\xi}}         % Xi vector
\providecommand{\vepsilon}{\boldsymbol{\epsilon}} % Epsilon vector

% =============================================================================
% MATRIX NOTATION (bold uppercase)
% =============================================================================

\providecommand{\mX}{\mathbf{X}}        % Data matrix
\providecommand{\mY}{\mathbf{Y}}        % Target matrix
\providecommand{\mW}{\mathbf{W}}        % Weight matrix
\providecommand{\mA}{\mathbf{A}}        % Generic matrix A
\providecommand{\mB}{\mathbf{B}}        % Generic matrix B
\providecommand{\mC}{\mathbf{C}}        % Generic matrix C
\providecommand{\mH}{\mathbf{H}}        % Hidden layer matrix / Hessian
\providecommand{\mI}{\mathbf{I}}        % Identity matrix
\providecommand{\mJ}{\mathbf{J}}        % Jacobian matrix
\providecommand{\mK}{\mathbf{K}}        % Kernel matrix
\providecommand{\mL}{\mathbf{L}}        % Loss matrix / Cholesky factor
\providecommand{\mP}{\mathbf{P}}        % Projection matrix
\providecommand{\mQ}{\mathbf{Q}}        % Orthogonal matrix
\providecommand{\mR}{\mathbf{R}}        % Rotation matrix
\providecommand{\mS}{\mathbf{S}}        % Scatter matrix
\providecommand{\mU}{\mathbf{U}}        % Left singular vectors
\providecommand{\mV}{\mathbf{V}}        % Right singular vectors

% Greek matrices (bold)
\providecommand{\mSigma}{\boldsymbol{\Sigma}}   % Covariance matrix
\providecommand{\mLambda}{\boldsymbol{\Lambda}} % Diagonal eigenvalue matrix
\providecommand{\mPhi}{\boldsymbol{\Phi}}       % Feature matrix
\providecommand{\mPsi}{\boldsymbol{\Psi}}       % Basis matrix
\providecommand{\mTheta}{\boldsymbol{\Theta}}   % Parameter matrix

% =============================================================================
% SETS AND SPACES (following Bishop/Murphy conventions)
% =============================================================================

\providecommand{\cD}{\mathcal{D}}       % Dataset
\providecommand{\cH}{\mathcal{H}}       % Hypothesis space
\providecommand{\cX}{\mathcal{X}}       % Input space
\providecommand{\cY}{\mathcal{Y}}       % Output space
\providecommand{\cF}{\mathcal{F}}       % Function space
\providecommand{\cG}{\mathcal{G}}       % Gaussian process
\providecommand{\cL}{\mathcal{L}}       % Lagrangian / Loss
\providecommand{\cN}{\mathcal{N}}       % Normal distribution
\providecommand{\cU}{\mathcal{U}}       % Uniform distribution
\providecommand{\cB}{\mathcal{B}}       % Bernoulli distribution
\providecommand{\cP}{\mathcal{P}}       % Probability distribution

% Number systems
\providecommand{\Real}{\mathbb{R}}      % Real numbers
\providecommand{\Nat}{\mathbb{N}}       % Natural numbers
\providecommand{\Int}{\mathbb{Z}}       % Integers
\providecommand{\Complex}{\mathbb{C}}   % Complex numbers

% =============================================================================
% OPERATORS AND FUNCTIONS (following standard conventions)
% =============================================================================

% Prediction notation (commonly used in ML)
\providecommand{\yhat}{\hat{\vy}}        % Predicted output vector (bold)
\providecommand{\yhati}{\hat{y}_i}       % Predicted output for sample i (scalar)

% Common ML functions (with conflict resolution)
\providecommand{\sigmoid}{}
\renewcommand{\sigmoid}{\operatorname{sigmoid}}
\providecommand{\softmax}{}
\renewcommand{\softmax}{\operatorname{softmax}}
\providecommand{\ReLU}{}
\renewcommand{\ReLU}{\operatorname{ReLU}}
\providecommand{\sign}{}
\renewcommand{\sign}{\operatorname{sign}}
\DeclareMathOperator{\Gain}{Gain}    % Information gain
\DeclareMathOperator{\Entropy}{Entropy}
% KL divergence (check for conflicts)
\providecommand{\KL}{}
\renewcommand{\KL}{\operatorname{KL}}
\DeclareMathOperator{\MSE}{MSE}      % Mean squared error
\DeclareMathOperator{\MAE}{MAE}      % Mean absolute error
\DeclareMathOperator{\RMSE}{RMSE}    % Root mean squared error

% Classification metrics (upright text)
\providecommand{\TP}{\text{TP}}          % True positives
\providecommand{\TN}{\text{TN}}          % True negatives  
\providecommand{\FP}{\text{FP}}          % False positives
\providecommand{\FN}{\text{FN}}          % False negatives
\DeclareMathOperator{\Precision}{Precision}
\DeclareMathOperator{\Recall}{Recall}
\DeclareMathOperator{\Accuracy}{Accuracy}

% Transpose and inverse
\providecommand{\tp}{^\top}             % Transpose (Bishop/Murphy style)
\providecommand{\inv}{^{-1}}            % Matrix inverse
\providecommand{\pinv}{^{\dagger}}      % Pseudoinverse

% Norms (consistent with Murphy/Bishop)
\providecommand{\norm}[1]{\|#1\|}       % Generic norm
\providecommand{\normone}[1]{\|#1\|_1}  % L1 norm
\providecommand{\normtwo}[1]{\|#1\|_2}  % L2 norm
\providecommand{\norminf}[1]{\|#1\|_\infty} % L-infinity norm
\providecommand{\normF}[1]{\|#1\|_F}    % Frobenius norm

% Optimization operators (upright as in Murphy)
\providecommand{\argmin}{}
\renewcommand{\argmin}{\operatorname*{arg\,min}}
\providecommand{\argmax}{}
\renewcommand{\argmax}{\operatorname*{arg\,max}}
\DeclareMathOperator{\minimize}{minimize}
\DeclareMathOperator{\maximize}{maximize}
\DeclareMathOperator{\subjectto}{subject\,to}

% Matrix operations (upright) - use conditional definitions
\providecommand{\tr}{}
\renewcommand{\tr}{\operatorname{tr}}       % Trace
\providecommand{\det}{}
\renewcommand{\det}{\operatorname{det}}     % Determinant
\providecommand{\rank}{}
\renewcommand{\rank}{\operatorname{rank}}   % Rank
\providecommand{\span}{}
\renewcommand{\span}{\operatorname{span}}   % Span
\providecommand{\null}{}
\renewcommand{\null}{\operatorname{null}}   % Null space
\DeclareMathOperator{\range}{range} % Range/column space
\providecommand{\diag}{}
\renewcommand{\diag}{\operatorname{diag}}   % Diagonal operator
\providecommand{\vec}{}
\renewcommand{\vec}{\operatorname{vec}}     % Vectorization operator

% Probability and statistics (Murphy/Bishop style)
\providecommand{\Prob}{\mathbb{P}}      % Probability measure
\providecommand{\Exp}{\mathbb{E}}       % Expectation
\DeclareMathOperator{\Var}{Var}     % Variance
\DeclareMathOperator{\Cov}{Cov}     % Covariance
\DeclareMathOperator{\Corr}{Corr}   % Correlation
% KL divergence already defined above
\DeclareMathOperator{\MI}{I}        % Mutual information

% Activation functions (already defined above with conflict resolution)

% =============================================================================
% STANDARD PARAMETER CONVENTIONS (Murphy/Bishop style)
% =============================================================================

% Primary parameters: θ (theta) - following Murphy's convention
% Learning rates: α, η (alpha, eta)
% Regularization: λ (lambda)
% Precision: β (beta) - following Bishop
% Variance: σ² (sigma squared)
% Standard deviation: σ (sigma)
% Mean: μ (mu)

% Common scalars:
% n - number of training examples
% d - dimensionality of input
% k - number of classes/clusters
% m - number of hidden units
% T - number of time steps
% i, j - indices

% =============================================================================
% STANDARD NOTATION EXAMPLES (Murphy/Bishop style)
% =============================================================================

% Linear regression:      y = \vw\tp\vx + b
% Matrix form:            \vy = \mX\vw + b\vone
% Logistic regression:    p(y=1|\vx) = \sigmoid(\vw\tp\vx)
% Gaussian:               \vx \sim \cN(\vmu, \mSigma)
% Parameter update:       \vtheta_{t+1} = \vtheta_t - \alpha \nabla \cL(\vtheta_t)
% Likelihood:             p(\cD|\vtheta) = \prod_{i=1}^n p(y_i|\vx_i, \vtheta)
% Posterior:              p(\vtheta|\cD) \propto p(\cD|\vtheta)p(\vtheta)
% Prediction:             p(y^*|\vx^*, \cD) = \int p(y^*|\vx^*, \vtheta)p(\vtheta|\cD)d\vtheta

% =============================================================================
% COMMON MISTAKES TO AVOID
% =============================================================================

% ❌ WRONG NOTATION          →  ✅ CORRECT NOTATION (Murphy/Bishop style)

% Transpose:
% ❌ x^t, X^t              →  ✅ \vx\tp, \mX\tp
% ❌ x', X'                →  ✅ \vx\tp, \mX\tp

% Vectors vs Matrices vs Scalars:
% ❌ X (for vector)        →  ✅ \vx (bold lowercase)
% ❌ w (for weight vector) →  ✅ \vw (bold lowercase)
% ❌ x (for data matrix)   →  ✅ \mX (bold uppercase)
% ❌ \mathbf{\theta}       →  ✅ \vtheta (Greek vectors are bold)
% ❌ \mathbf{n}            →  ✅ n (scalars are not bold)

% Sets and distributions:
% ❌ R                     →  ✅ \Real (blackboard bold for number systems)
% ❌ \mathcal{R}           →  ✅ \Real (use blackboard for reals)
% ❌ Normal               →  ✅ \cN (calligraphic for distributions)

% Functions and operators:
% ❌ argmax                →  ✅ \argmax (upright operator)
% ❌ E[X]                  →  ✅ \Exp[X] (blackboard E for expectation)
% ❌ trace(A)              →  ✅ \tr(\mA) (upright operator)

% =============================================================================
% ALGORITHM NAME CONVENTIONS
% =============================================================================

% Use standard capitalizations as in textbooks:
% k-NN, SVM, PCA, GMM, EM, MAP, ML, SGD, Adam, RMSprop
% ReLU, tanh, sigmoid, softmax

\endinput