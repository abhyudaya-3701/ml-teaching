\documentclass{beamer}
\usepackage{../../shared/styles/custom}
\usepackage{../../shared/styles/conventions}
\usepackage{../../shared/notation/notation}

\mathtoolsset{showonlyrefs}  
\title{Multivariate Normal Distribution II}
\date{\today}
\author{Nipun Batra and teaching staff}
\institute{IIT Gandhinagar}
\begin{document}
	\maketitle
	
	\urldef\nodeSix\url{http://fourier.eng.hmc.edu/e161/lectures/gaussianprocess/node6.html}
	\urldef\nodeSeven\url{http://fourier.eng.hmc.edu/e161/lectures/gaussianprocess/node7.html}
	
	
	\begin{frame}{Detour: Inverse of partioned symmetric matrix \footnote{Courtesy: \nodeSix}}
		
		Consider an $n\times n$ symmetric matrix $\mA$ and divide it into four blocks
		
		$
		\mA = \begin{bmatrix}
		\mA_{11} & \mA_{12}\\
		\mA_{21} & \mA_{22} \\
		\end{bmatrix} =  \begin{bmatrix}
		\mA_{11} & \mA_{12}\\
		\mA_{12}\tp & \mA_{22} \\
		\end{bmatrix}
		$
		
		For example, let $n=3$, we have
		
		$$
		\mA =  \begin{bmatrix}
		1 & 2 & 3\\
		2 & 5 & 6 \\
		3 & 6 & 8 \\
		
	\end{bmatrix}
	$$
	
	We could for example have 
	
	$\mA_{11} = \begin{bmatrix}
	1 & 2 \\
	2 & 5 \\
	\end{bmatrix}$ and $\mA_{12} = \begin{bmatrix}
	3 \\ 6
	\end{bmatrix}
	$ and $\mA_{22} = \begin{bmatrix}
	8
	\end{bmatrix}$
\end{frame}

\begin{frame}{Detour: Inverse of partioned symmetric matrix}
	Question: Write $\mB = \mA\inv$ in terms of the four blocks
	$$
	\mB = \begin{bmatrix}
	\mB_{11} & \mB_{12}\\
	\mB_{12}\tp & \mB_{22} \\
	\end{bmatrix}
	= \mA\inv$$

	$\mA_{11}$ and $\mB_{11} \in \Real^{p\times p}$ \\
	$\mA_{22}$ and $\mB_{22} \in \Real^{q\times q}$ \\
	$\mA_{12} = \mA_{21}\tp$ and $\mB_{12} = \mB_{21}\tp \in \Real^{p\times q}$ \\
	and, p + q = n 
\end{frame}

\begin{frame}{Detour: Inverse of partioned symmetric matrix}
	$\mI_n = \mA\mA\inv = \mA\mB$
	
	$=\begin{bmatrix}
	\mA_{11} & \mA_{12}\\
	\mA_{12}\tp & \mA_{22} \\
	\end{bmatrix} \begin{bmatrix}
	\mB_{11} & \mB_{12}\\
	\mB_{12}\tp & \mB_{22} \\
	\end{bmatrix}
	= \begin{bmatrix}
	\mA_{11}\mB_{11} +\mA_{12}\mB_{12}\tp & \mA_{11}\mB_{12} + \mA_{12}\mB_{22}\\
	\mA_{12}\tp\mB_{11} + \mA_{22}\mB_{12}\tp & \mA_{12}\tp\mB_{12} + \mA_{22}\mB_{22}  \\
	\end{bmatrix} = \begin{bmatrix}
	\mI_p & \vzero \\
	\vzero & \mI_q
	\end{bmatrix}$
	
	Thus, we have
	\begin{gather}
	\mA_{11}\mB_{11} +\mA_{12}\mB_{12}\tp = \mI_p\\
	\mA_{11}\mB_{12} + \mA_{12}\mB_{22} = \vzero^{p\times q}\\
	\mA_{12}\tp\mB_{11} + \mA_{22}\mB_{12}\tp = \vzero^{q\times p}\\
	\mA_{12}\tp\mB_{12} + \mA_{22}\mB_{22} = \mI_q
	\end{gather}
\end{frame}

\begin{frame}{Detour: Inverse of partioned symmetric matrix}
	Moving the expressions around we get the following results.
	\begin{align*}
	\mB_{11} &= (\mA_{11} - \mA_{12}\mA_{22}\inv \mA_{12}\tp)\inv \\
	&= \mA_{11}\inv + \mA_{11}\inv \mA_{12}(\mA_{22} - \mA_{12}\tp\mA_{11}\inv \mA_{12})\inv \mA_{12}\tp\mA_{11}\inv\\
	\mB_{22} &= (\mA_{22} - \mA_{12}\tp\mA_{11}\inv \mA_{12})\inv \\
	&= \mA_{22}\inv + \mA_{22}\inv \mA_{12}\tp(\mA_{11} - \mA_{12}\mA_{22}\inv \mA_{12}\tp)\inv \mA_{12}\mA_{22}\inv\\
	\mB_{12}\tp &= - \mA_{22}\inv \mA_{12}\tp ( \mA_{11} - \mA_{12} \mA_{22}\inv \mA_{12}\tp)\inv\\
	\mB_{12}\tp &= - \mA_{11}\inv \mA_{12}\tp ( \mA_{22} - \mA_{12}\tp \mA_{11}\inv \mA_{12})\inv
	\end{align*}
\end{frame}

\begin{frame}{Determinant of Partitioned Symmetric Matrix}
	
	\textbf{Theorem:} Determinant of a partitioned symmetric matrix can be written as follows
	\begin{gather}
	\det(\mA) = \det \left( 
	\begin{matrix}
	\mA_{11}&\mA_{12}\\
	\mA_{21}&\mA_{22}
	\end{matrix}
	\right) \\
	=\det(\mA_{11})\det(\mA_{22}-\mA_{12}\tp\mA_{11}\inv\mA_{12}) \\
	=\det(\mA_{22})\det(\mA_{11}-\mA_{12}\mA_{22}\inv\mA_{12}\tp)
	\end{gather}
\end{frame}

\begin{frame}{Determinant of Partitioned Symmetric Matrix}
	\textbf{Proof:} Note that
	\begin{gather}
	\mA = 
	\begin{bmatrix}
	\mA_{11}&\mA_{12} \\
	\mA_{21}&\mA_{22}
	\end{bmatrix}
	= 
	\begin{bmatrix}
	\mA_{11}& \vzero \\
	\mA_{12}\tp& \mI
	\end{bmatrix}
	\begin{bmatrix}
	\mI & \mA_{11}\inv\mA_{12} \\
	\vzero & \mA_{22} - \mA_{12}\tp\mA_{11}\inv\mA_{12}
	\end{bmatrix}\\
	= 
	\begin{bmatrix}
	\mI & \mA_{12} \\
	\vzero & \mA_{22}
	\end{bmatrix}
	\begin{bmatrix}
	\mA_{11} - \mA_{12}\mA_{22}\inv\mA_{12}\tp & \vzero\\
	\mA_{22}\inv\mA_{21}  & \mI
	\end{bmatrix}
	\end{gather} 
	
	The theorem is proved as we also know that
	\begin{equation}
	\det(\mA\mB)=\det(\mA) \det(\mB) 
	\end{equation}
	
	and
	\begin{equation}
	\det \left( 
	\begin{matrix}
	\mB&\vzero \\
	\mC&\mD
	\end{matrix} 
	\right)
	=
	\det \left(
	\begin{matrix}
	\mB&\mC \\
	\vzero&\mD
	\end{matrix}
	\right)
	=
	\det(\mB)\det(\mD)
	\end{equation}	
\end{frame}

\begin{frame}{Marginalisation and Conditional of multivariate normal\footnote{Courtesy: \nodeSeven.}}
	Assume an n-dimensional random vector
	
	\begin{equation}
	{\vx}=\begin{bmatrix}\vx_1 & \vx_2\end{bmatrix} 
	\end{equation}
	
	has a normal distribution $\distribReal{N}{\vmu}{\mSigma}$ with
	\begin{gather}\vmu=
	\begin{bmatrix}
	\vmu_1 \\
	\vmu_2
	\end{bmatrix} 
	\text{and }
	\mSigma = \begin{bmatrix}
	\mSigma_{11}& \mSigma_{12}\\
	\mSigma_{21}&\mSigma_{22}
	\end{bmatrix} 
	\end{gather}

	where $\vx_1$ and $\vx_2$ are two subvectors of respective dimensions $p$ and $q$ with $p+q=n$. Note that $\mSigma=\mSigma\tp$, and $\mSigma_{21}=\mSigma_{21}\tp$.
\end{frame}

\begin{frame}{Marginalisation and Conditional of multivariate normal}
	\textbf{Theorem:}
	
	\textbf{part a:} The marginal distributions of $\vx_1$ and $\vx_2$ are also normal with mean vector $\vmu_i$ and covariance matrix $\mSigma_{ii}$ ($i=1,2$), respectively.
	
	\textbf{part b:} The conditional distribution of $\vx_i$ given $\vx_j$ is also normal with mean vector
	\begin{gather}
	\vmu_{i|j} = \vmu_i + \mSigma_{ij}\mSigma_{jj}^{-1}(\vx_j - \vmu_j)\\
	\end{gather}
\end{frame}

\begin{frame}{Marginalisation and Conditional of multivariate normal}
	\textbf{Proof:}
	
	The joint density of ${\vx}$ is:
	
	\begin{equation}
	f({\vx})=f(\vx_1,\vx_2)=\frac{1}{(2\pi)^{n/2}\det(\mSigma)^{1/2}}exp\left[-\frac{1}{2}Q(\vx_1,\vx_2)\right] 
	\end{equation}
	
	where $Q$ is defined as
    \small
	\begin{gather}
	Q(\vx_1,\vx_2) = ({\vx}-\vmu)\tp\mSigma\inv({\vx}-\vmu)\\
	= [(\vx_1-\vmu_1)\tp, (\vx_2-\vmu_2)\tp] 
	\begin{bmatrix}
	\mSigma^{11} & \mSigma^{12}\\
	\mSigma^{21} & \mSigma^{22}
	\end{bmatrix}
	\begin{bmatrix}
	\vx_1-\vmu_1 \\
	\vx_2-\vmu_2
	\end{bmatrix}\\
	= (\vx_1-\vmu_1)\tp\mSigma^{11}(\vx_1-\vmu_1)+ 2(\vx_1-\vmu_1)\tp\mSigma^{12}(\vx_2-\vmu_2) + (\vx_2-\vmu_2)\tp \cdots \\
	\cdots \mSigma^{22}(\vx_2-\vmu_2)
	\end{gather}
\end{frame}

\begin{frame}{Marginalisation and Conditional of multivariate normal}
Here we have assumed 
$$	
\mSigma\inv=\left[\begin{array}{cc}
{\mSigma_{11}} & {\mSigma_{12}} \\
{\mSigma_{21}} & {\mSigma_{22}}
\end{array}\right]\inv=\left[\begin{array}{cc}
{\mSigma^{11}} & {\mSigma^{12}} \\
{\mSigma^{21}} & {\mSigma^{22}}
\end{array}\right]
$$
According to inverse of a partitioned symmetric matrix we have, 
\begin{align*}
\mSigma^{11}&=\left(\mSigma_{11}-\mSigma_{12} \mSigma_{22}\inv \mSigma_{12}\tp\right)\inv \\
&=\mSigma_{11}\inv+\mSigma_{11}\inv \mSigma_{12}\left(\mSigma_{22}-\mA_{12}\tp \mSigma_{11}\tp \mSigma_{12}\right)\inv \mSigma_{12}\tp \mSigma_{11}\inv \\
\mSigma^{22}&=\left(\mSigma_{22}-\mSigma_{12}\tp \mSigma_{11}\inv \mSigma_{12}\right)\inv \\
&=\mSigma_{22}\inv+\mSigma_{22}\inv \mSigma_{12}\tp\left(\mSigma_{11}-\mSigma_{12} \mSigma_{22}\inv \mSigma_{12}\tp\right)\inv \mSigma_{12} \mSigma_{22}\inv \\
\mSigma^{12}&=-\mSigma_{11}\inv \mSigma_{12}\left(\mSigma_{22}-\mSigma_{12}\tp \mSigma_{11}\inv \mSigma_{12}\right)\inv=\left(\mSigma^{21}\right)\tp
\end{align*}
	
\end{frame}

\begin{frame}{Marginalisation and Conditional of multivariate normal}
	Substituting the second expression for $\mSigma^{11}$, first expression for $\mSigma^{22}$, and $\mSigma^{12}$ into $Q(\vx_1,\vx_2)$ to get:
	\footnotesize
	\begin{gather}
	\begin{aligned}
	&Q\left(\vx_{1}, \vx_{2}\right)= \\
    &\left(\vx_{1}-\vmu_{1}\right)\tp\left[\mSigma_{11}\inv+\mSigma_{11}\inv \mSigma_{12}\left(\mSigma_{22}-\mA_{12}\tp \mSigma_{11}\inv \mSigma_{12}\right)\inv \mSigma_{12}\tp \mSigma_{11}\inv\right]\left(\vx_{1}-\vmu_{1}\right) \\
	&-2\left(\vx_{1}-\vmu_{1}\right)\tp\left[\mSigma_{11}\inv \mSigma_{12}\left(\mSigma_{22}-\mSigma_{12}\tp \mSigma_{11}\inv \mSigma_{12}\right)\inv\right]\left(\vx_{2}-\vmu_{2}\right) \\
	&+\left(\vx_{2}-\vmu_{2}\right)\tp\left[\left(\mSigma_{22}-\mSigma_{12}\tp \mSigma_{11}\inv \mSigma_{12}\right)\inv\right]\left(\vx_{2}-\vmu_{2}\right)\\
	=&\left(\vx_{1}-\vmu_{1}\right)\tp \mSigma_{11}\inv\left(\vx_{1}-\vmu_{1}\right) \\
	&\left.+\left(\vx_{1}-\vmu_{1}\right)\tp \mSigma_{11}\inv \mSigma_{12}\left(\mSigma_{22}-\mA_{12}\tp \mSigma_{11}\inv \mSigma_{12}\right)\inv \mSigma_{12}\tp \mSigma_{11}\inv\right]\left(\vx_{1}-\vmu_{1}\right) \\
	&-2\left(\vx_{1}-\vmu_{1}\right)\tp\left[\mSigma_{11}\inv \mSigma_{12}\left(\mSigma_{22}-\mSigma_{12}\tp \mSigma_{11}\inv \mSigma_{12}\right)\inv\right]\left(\vx_{2}-\vmu_{2}\right) \\
	&+\left(\vx_{2}-\vmu_{2}\right)\tp\left[\left(\mSigma_{22}-\mSigma_{12}\tp \mSigma_{11}\inv \mSigma_{12}\right)\inv\right]\left(\vx_{2}-\vmu_{2}\right)
	\end{aligned}\\
	\end{gather}
\end{frame}

\begin{frame}{Marginalisation and Conditional of multivariate normal}
	\footnotesize
	\begin{gather}
	\begin{aligned}
	=&\left(\vx_{1}-\vmu_{1}\right)\tp \mSigma_{11}\inv\\
	&+\left[\left(\vx_{2}-\vmu_{2}\right)-\mSigma_{12}\tp \mSigma_{11}\inv\left(\vx_{1}-\vmu_{1}\right)\right]\tp \cdot \left(\mSigma_{22}-\mSigma_{12}\tp \mSigma_{11}\inv \mSigma_{12}\right)\inv \\
    & \cdot \left[\left(\vx_{2}-\vmu_{2}\right)-\mSigma_{12}\tp \mSigma_{11}\inv\left(\vx_{1}-\vmu_{1}\right)\right]
	\end{aligned}
	\end{gather}
	\normalsize
	The last equal sign is due to the following equations for any vectors $\vu$ and $\vv$ and a symmetric matrix $\mA=\mA\tp$:
	
	\begin{gather}
	\begin{aligned}
	& \vu\tp \mA \vu-2 \vu\tp \mA \vv+\vv\tp \mA \vv=\vu\tp \mA \vu-\vu\tp \mA \vv-\vu\tp \mA \vv+\vv\tp \mA \vv \\
	=& \vu\tp \mA (\vu-\vv)-(\vu-\vv)\tp \mA \vv=\vu\tp \mA (\vu-\vv)-\vv\tp \mA (\vu-\vv) \\
	=&(\vu-\vv)\tp \mA (\vu-\vv)=(\vv-\vu)\tp \mA (\vv-\vu)
	\end{aligned}
	\end{gather}
\end{frame}

\begin{frame}{Marginalisation and Conditional of multivariate normal}
	We define  \\
	$\vb \triangleq \vmu_{2}+\mSigma_{12}\tp \mSigma_{11}\inv\left(\vx_{1}-\vmu_{1}\right)$
	\[
	\mA \triangleq \mSigma_{22}-\mSigma_{12}\tp \mSigma_{11}\inv \mSigma_{12}
	\]
	
	and 

	$$\left\{\begin{array}{l}{Q_{1}\left(\vx_{1}\right) \quad \triangleq\left(\vx_{1}-\vmu_{1}\right)\tp \mSigma_{1}\inv\left(\vx_{1}-\vmu_{1}\right)} \\ 
		{\mN=\left[\left(\vx_{2}-\vmu_{2}\right)-\mSigma_{12}\tp \mSigma_{11}\inv\left(\vx_{1}-\vmu_{1}\right)\right]} \\
		{Q_{2}\left(\vx_{1}, \vx_{2}\right) \triangleq \mN\tp\left(\mSigma_{22}-\mSigma_{12}\tp \mSigma_{11}\inv \mSigma_{12}\right)\inv \mN} \\
		{=\left(\vx_{2}-\vb\right)\tp \mA\inv\left(\vx_{2}-\vb\right)}\end{array}\right.$$

	and get 
	$$Q\left(\vx_{1}, \vx_{2}\right)=Q_{1}\left(\vx_{1}\right)+Q_{2}\left(\vx_{1}, \vx_{2}\right)$$
\end{frame}


\begin{frame}{Marginalisation and Conditional of multivariate normal}
	\small
    Now the joint distribution can be written as: 
	
	$$
	\begin{aligned}
		f(\vx) &=f\left(\vx_{1}, \vx_{2}\right)=\frac{1}{(2 \pi)^{n / 2}\det(\mSigma)^{1 / 2}} \exp \left[-\frac{1}{2} Q\left(\vx_{1}, \vx_{2}\right)\right] \\ 
		&=\frac{1}{(2 \pi)^{n / 2}\det(\mSigma_{11})^{1 / 2}\det(\mSigma_{22}-\mSigma_{12}\tp \mSigma_{11}\inv \mSigma_{12})^{1 / 2}} \exp \left[-\frac{1}{2} Q\left(\vx_{1}, \vx_{2}\right)\right] \\ 
		&=\frac{1}{(2 \pi)^{p / 2}\det(\mSigma_{11})^{1 / 2}} \exp \left[-\frac{1}{2}\left(\vx_{1}-\vmu_{1}\right)\tp \mSigma_{11}\inv\left(\vx_{1}-\vmu_{1}\right)\right] \\
		&\quad \times \frac{1}{(2 \pi)^{q / 2}\det(\mA)^{1 / 2}} \exp \left[-\frac{1}{2}\left(\vx_{2}-\vb\right)\tp \mA\inv\left(\vx_{2}-\vb\right)\right] \\ 
		&=\distribReal{N}{\vmu_1}{\mSigma_{11}}_{\vx_1}\ \distribReal{N}{\vb}{\mA}_{\vx_2}
	\end{aligned}
	$$
	
	The third equal sign is due to Determinant of a partitioned symmetric matrix:

	$$
	\det(\mSigma)=\det(\mSigma_{11})\det(\mSigma_{22}-\mSigma_{12}\tp \mSigma_{11}\inv \mSigma_{12})
	$$
\end{frame}

\begin{frame}{Marginalisation and Conditional of multivariate normal}
	The marginal distribution of $\vx_1$ is 
	\begin{align*}
    f_{1}\left(\vx_{1}\right)&=\int f\left(\vx_{1}, \vx_{2}\right) d \vx_{2} \\
    &=\frac{1}{(2 \pi)^{p / 2}\det(\mSigma_{11})^{1 / 2}} \exp \left[-\frac{1}{2}\left(\vx_{1}-\vmu_{1}\right)\tp \mSigma_{11}\inv\left(\vx_{1}-\vmu_{1}\right)\right]
	\end{align*}
	and the conditional distribution of $\vx_2$ given $\vx_1$ is 
	\begin{align*}
	f_{2 | 1}\left(\vx_{2} | \vx_{1}\right)&=\frac{f\left(\vx_{1}, \vx_{2}\right)}{f\left(\vx_{1}\right)} \\
	&=\frac{1}{(2 \pi)^{q / 2}\det(\mA)^{1 / 2}} \exp \left[-\frac{1}{2}\left(\vx_{2}-\vb\right)\tp \mA\inv\left(\vx_{2}-\vb\right)\right]
	\end{align*}
\end{frame}

\begin{frame}{Marginalisation and Conditional of multivariate normal}
	with
	
	$$\vb=\vmu_{2}+\mSigma_{12}\tp \mSigma_{11}\inv\left(\vx_{1}-\vmu_{1}\right)$$
	\[
	\mA=\mSigma_{22}-\mSigma_{12}\tp \mSigma_{11}\inv \mSigma_{12}
	\]
\end{frame}


%\begin{frame}{Marginalisation of bivariate normal}
%	Complete derivation from \textcolor{red}{Where?}
%\end{frame}
%
%\begin{frame}{Conditional Normal Distribution}
%
%\end{frame}
\end{document}