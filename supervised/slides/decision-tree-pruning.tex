\documentclass[usenames,dvipsnames]{beamer}
\usepackage{../../shared/styles/custom}
\usepackage{../../shared/styles/conventions}

\title{Decision Trees}
\date{\today}
\author{Nipun Batra and teaching staff}
\institute{IIT Gandhinagar}

\begin{document}

\section{Pruning and Overfitting}

\begin{frame}{The Problem: Overfitting in Decision Trees}
\cleanitemize{
    \item \textbf{Unpruned trees}: Can grow very deep and complex
    \item \textbf{Perfect training accuracy}: Each leaf contains single training example
    \item \textbf{But}: Poor generalization to new data
    \item \textbf{Symptoms}:
    \cleanitemize{
        \item High training accuracy, low test accuracy
        \item Very deep trees with many leaves
        \item Rules that are too specific to training data
    }
    \item \textbf{Solution}: Pruning to control model complexity
}
\end{frame}

\begin{frame}{Pre-pruning (Early Stopping)}
\textbf{Stop growing tree before it becomes too complex}:
\cleanitemize{
    \item \textbf{Maximum depth}: Limit tree depth (e.g., max\_depth = 5)
    \item \textbf{Minimum samples per split}: Don't split if node has < N samples
    \item \textbf{Minimum samples per leaf}: Ensure each leaf has $\geq$ M samples
    \item \textbf{Maximum features}: Consider only subset of features at each split
    \item \textbf{Minimum impurity decrease}: Only split if improvement > threshold
}
\textbf{Advantages}: Simple, computationally efficient \\
\textbf{Disadvantages}: May stop too early, miss good splits later
\end{frame}

\begin{frame}{Post-pruning (Tree Simplification)}
\textbf{Grow full tree, then remove unnecessary branches}:
\cleanitemize{
    \item \textbf{Algorithm}:
        \begin{enumerate}
            \item Grow complete tree on training data
            \item Use validation set to evaluate subtree performance
            \item Remove branches that don't improve validation accuracy
            \item Repeat until no beneficial removals remain
        \end{enumerate}
    \item \textbf{Cost Complexity Pruning}: Minimize $\text{Error} + \alpha \times \text{Tree Size}$
    \item \textbf{Advantages}: More thorough, can recover from early stopping mistakes
    \item \textbf{Disadvantages}: More computationally expensive
}
\end{frame}

\begin{frame}{Cost Complexity Pruning Algorithm}
\textbf{Systematic approach to find optimal tree size}:
\cleanitemize{
    \item \textbf{Cost function}: $R_\alpha(T) = R(T) + \alpha |T|$
        \cleanitemize{
            \item $R(T)$: Total impurity (training error)
            \item $|T|$: Number of leaves
            \item $\alpha$: Complexity penalty parameter
        }
    \item \textbf{Weakest Link}: At each pruning step, compute:
        $$g(t) = \frac{R(t) - R(T_t)}{|T_t| - 1}$$
        \cleanitemize{
            \item $g(t)$: The $\alpha$ value at which subtree rooted at node $t$ should be pruned
            \item $R(t)$: Impurity of node $t$, treating it as leaf node
            \item $R(T_t)$: Total impurity of subtree rooted at node $t$
            \item $|T_t|$: Number of leaves in subtree rooted at node $t$
        }
}
\end{frame}

\begin{frame}{Cost Complexity Pruning: Algorithm Steps}
\textbf{Iterative pruning process}:
\cleanitemize{
    \item \textbf{Process}:
        \begin{enumerate}
            \item Start with full tree ($\alpha = 0$)
            \item Compute $g(t)$ for all internal nodes
            \item Prune node with smallest $g(t)$ (weakest link)
            \item Repeat until only root remains
            \item Use cross-validation to select optimal $\alpha$
        \end{enumerate}
}
\end{frame}

\begin{frame}{Bias-Variance Trade-off in Trees}
\cleanitemize{
    \item \textbf{Unpruned trees}:
        \cleanitemize{
            \item Low bias (can fit complex patterns)
            \item High variance (sensitive to training data changes)
            \item Prone to overfitting
        }
    \item \textbf{Heavily pruned trees}:
        \cleanitemize{
            \item High bias (may miss important patterns)
            \item Low variance (more stable predictions)
            \item Risk of underfitting
        }
    \item \textbf{Optimal pruning}: Balances bias and variance
    \item \textbf{Cross-validation}: Essential for finding this balance
}
\end{frame}

\begin{frame}{Practical Pruning Guidelines}
\cleanitemize{
    \item \textbf{Start simple}: Begin with restrictive pre-pruning parameters
    \item \textbf{Cross-validation}: Always use CV to select pruning parameters
    \item \textbf{Validation curves}: Plot training/validation error vs. tree complexity
    \item \textbf{Common parameters} (sklearn):
        \cleanitemize{
            \item \texttt{max\_depth}: Start with 3-10
            \item \texttt{min\_samples\_split}: Try 10-100
            \item \texttt{min\_samples\_leaf}: Try 5-50
            \item \texttt{ccp\_alpha}: Use for cost complexity pruning
        }
    \item \textbf{Domain knowledge}: Consider interpretability requirements
}
\end{frame}

\end{document}