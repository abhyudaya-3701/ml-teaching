\documentclass{beamer}
\usepackage{../../shared/styles/custom}
\usepackage{../../shared/styles/conventions}


%\beamerdefaultoverlayspecification{<+->}
% \newcommand{\data}{\mathcal{D}}
% \newcommand\Item[1][]{%
% 	\ifx\relax#1\relax  \item \else \item[#1] \fi
% 	\abovedisplayskip=0pt\abovedisplayshortskip=0pt~\vspace*{-\baselineskip}}

\graphicspath{ {../assets/bias-variance/figures/} }



\title{Bias-Variance and Cross Validation}
\date{\today}
\author{Nipun Batra and teaching staff}
\institute{IIT Gandhinagar}
\begin{document}
\maketitle

\begin{frame}{Table of Contents}
\tableofcontents
\end{frame}

\section{Introduction to Bias-Variance}

\begin{frame}{A Question!}
	What would be the decision boundary of a decision tree classifier? 
		
	\begin{figure}
		\centering
	\includegraphics[scale=0.55]{dataset-1}
	\end{figure}


	\end{frame}

	\begin{frame}{Decision Boundary for a tree with depth 1}
	\begin{figure}[h]
	    \centering
	    \begin{minipage}{0.45\textwidth}
	        \centering
	        \includegraphics[width=\textwidth]{example-1-depth-1-boundary}
	        \caption{Decision Boundary}
	    \end{minipage}
	    \hfill
	    \begin{minipage}{0.45\textwidth}
	        \centering
	        \includegraphics[width=\textwidth]{example-1-depth-1-decision-tree}
	        \caption{Decision Tree}
	    \end{minipage}
	\end{figure}
	\end{frame}
	
	\begin{frame}{Decision Boundary for a tree with no depth limit}
	\begin{figure}[h]
	    \centering
	    \begin{minipage}{0.45\textwidth}
	        \centering
	        \includegraphics[width=\textwidth]{example-1-nolimit-boundary}
	        \caption{Decision Boundary}
	    \end{minipage}
	    \hfill
	    \begin{minipage}{0.45\textwidth}
	        \centering
	        \includegraphics[width=\textwidth]{example-1-nolimit-decision-tree}
	        \caption{Decision Tree}
	    \end{minipage}
	\end{figure}
	\end{frame}
	

	\begin{frame}{Are deeper trees always better?}
	\only<1-2>{
	As we saw, deeper trees learn more complex decision boundaries.
	}
	
	\only<2>{
	\vspace{1cm}
	But, sometimes this can lead to poor generalization
	}	
	\end{frame}

	\begin{frame}{An example}
	Consider the dataset below
	\begin{figure}[h]
	    \centering
	    \begin{minipage}{0.45\textwidth}
	        \centering
	        \includegraphics[width=\textwidth]{dataset-2-train}
	        \caption{Train Set}
	    \end{minipage}
	    \hfill
	    \begin{minipage}{0.45\textwidth}
	        \centering
	        \includegraphics[width=\textwidth]{dataset-2-test}
	        \caption{Test Set}
	    \end{minipage}
	\end{figure}
	\end{frame}

	\begin{frame}{Underfitting}
	Underfitting is also known as high bias, since it has a very biased incorrect assumption.
	\begin{figure}[h]
	    \centering
	    \begin{minipage}{0.45\textwidth}
	        \centering
	        \includegraphics[width=\textwidth]{example-2-depth-1-boundary}
	        \caption{Decision Boundary}
	    \end{minipage}
	    \hfill
	    \begin{minipage}{0.45\textwidth}
	        \centering
	        \includegraphics[width=\textwidth]{example-2-depth-1-decision-tree}
	        \caption{Decision Tree}
	    \end{minipage}
	\end{figure}
	\end{frame}

	\begin{frame}{Overfitting}
	Overfitting is also known as high variance, since very small changes in data can lead to very different models.\\
	Decision tree learned has depth of 10.
	\begin{center}
	\includegraphics[scale=0.5]{example-2-nolimit-boundary}
	\end{center}
	\end{frame}


	\begin{frame}{Intuition for Variance}
	A small change in data can lead to very different models.\\
	\vspace{1cm}
	\begin{columns}
		\begin{column}{0.5\textwidth}{\hspace{1.75cm} Dataset 1}
			\begin{center}
			\includegraphics[width = \textwidth]{dataset-2-train}
			\end{center}
		\end{column}
		\begin{column}{0.5\textwidth}{\hspace{1.75cm} Dataset 2}
			\begin{center}
			\includegraphics[width = \textwidth]{dataset-2-train-var}
			\end{center}
		\end{column}
	\end{columns}
	\end{frame}


	\begin{frame}{Intuition for Variance}
	\begin{columns}
		\begin{column}{0.5\textwidth}
			\begin{center}
			\includegraphics[scale=0.2]{var_1}
			\end{center}
		\end{column}
		\begin{column}{0.5\textwidth}
			\begin{center}
			\includegraphics[scale=0.2]{var_2}
			\end{center}
		\end{column}
	\end{columns}
	\end{frame}


	\begin{frame}{A Good Fit}
	\begin{columns}
		\begin{column}{0.5\textwidth}
			\includegraphics[width=\textwidth]{example-2-optimal-boundary}
		\end{column}
		\begin{column}{0.5\textwidth}
			\includegraphics[width =\textwidth]{example-2-optimal-tree}
		\end{column}
	\end{columns}

	\end{frame}

	\begin{frame}{Accuracy vs Depth Curve}
		\begin{center}
		\includegraphics[scale=0.55]{acc-depth-plot}
	\end{center}
	\pause As depth increases, train accuracy improves\\
	\pause As depth increases, test accuracy improves till a point\\
	\pause At very high depths, test accuracy is not good (overfitting). 

	\end{frame}

	\begin{frame}{Accuracy vs Depth Curve : Underfitting}
	The highlighted region is the underfitting region.\\
	Model is too simple (less depth) to learn from the data. 
	\begin{center}
	\includegraphics[scale=0.55]{acc-depth-plot-underfit}
	\end{center}
	\end{frame}

	\begin{frame}{Accuracy vs Depth Curve : Overfitting}
	The highlighted region is the overfitting region.\\
	Model is complex (high depth) and hence also learns the anomalies in data. 
	\begin{center}
	\includegraphics[scale=0.55]{acc-depth-plot-overfit}
	\end{center}
	\end{frame}

	\begin{frame}{Accuracy vs Depth Curve }
	The highlighted region is the good fit region.\\
	We want to maximize test accuracy while being in this region.
	\begin{center}
	\includegraphics[scale=0.55]{acc-depth-plot-properfit}
	\end{center}
	\end{frame}

	\begin{frame}{The big question!?}
	\only<1-2>{
	How to find the optimal depth for a decision tree?\\
	}
	\only<2>{
	\vspace{1cm}
	Use cross-validation!
	}
	\end{frame}


	\begin{frame}{Our General Training Flow}
	\includegraphics[width = \textwidth]{../assets/bias-variance/diagrams/general-workflow}
	\end{frame}

	\begin{frame}{K-Fold cross-validation: Utilise full dataset for testing}
	\includegraphics[width = \textwidth]{../assets/bias-variance/diagrams/cross-validation-train-test.pdf}
	\end{frame}

	\begin{frame}{The Validation Set}
	\includegraphics[width = \textwidth]{../assets/bias-variance/diagrams/validation-workflow}
	\end{frame}

	\begin{frame}{Nested Cross Validation}
	Divide your training set into $k$ equal parts.\\
	 Cyclically use 1 part as ``validation set'' and the rest for training.\\
	Here $k = 4$
	\begin{center}
	\includegraphics[scale=0.7]{../assets/bias-variance/diagrams/cross-validation}
	\end{center}
	\end{frame}

	\begin{frame}{Nested Cross Validation}
	Average out the validation accuracy across all the folds\\
	Use the model with highest validation accuracy\\
	\includegraphics[width = \textwidth]{../assets/bias-variance/diagrams/cross-validation-avg}
	\end{frame}

\section{Practice and Review}

\begin{frame}{Pop Quiz: Bias-Variance Concepts}
\begin{enumerate}
\item What causes high bias in a model? Give an example.
\pause
\item What causes high variance in a model? Give an example.
\pause
\item How does cross-validation help in model selection?
\pause
\item Why can't we directly optimize for test error?
\end{enumerate}
\end{frame}

\begin{frame}{Key Takeaways}
\begin{itemize}[<+->]
\item \textbf{Bias-Variance Decomposition}: Total error = Bias² + Variance + Noise
\item \textbf{High Bias}: Underfitting, model too simple
\item \textbf{High Variance}: Overfitting, model too complex
\item \textbf{Cross-Validation}: Essential for proper model evaluation
\item \textbf{Model Selection}: Choose complexity that balances bias and variance
\item \textbf{No Free Lunch}: Cannot reduce both bias and variance simultaneously
\end{itemize}
\end{frame}

\begin{frame}{Next time: Ensemble Learning}
\begin{itemize}
	\item How to combine various models?
	\item Why to combine multiple models?
	\item How can we reduce bias?
	\item How can we reduce variance?
\end{itemize}
\end{frame}

\end{document}
