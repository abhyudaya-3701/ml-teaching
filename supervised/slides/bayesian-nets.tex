\documentclass{beamer}
\usepackage{../../shared/styles/custom}
\usepackage{../../shared/styles/conventions}

%\beamerdefaultoverlayspecification{<+->}

	\ifx\relax#1\relax  \item \else \item[#1] \fi
	\abovedisplayskip=0pt\abovedisplayshortskip=0pt~\vspace*{-\baselineskip}}

\title{Naive Bayes}
\date{\today}
\author{Nipun Batra}
\institute{IIT Gandhinagar}
\begin{document}
  \maketitle

% \section{Linear Regression}

\begin{frame}\begin{equation*}
        P(x_{1},x_{2},x_{3},\dots,x_{N} \vert y) = P(x_{1}|y) P(x_{2}|y) \dots P(x_{N}|y)
    \end{equation*}
    \pause Why is Naive Bayes model called Naive? \\

    \pause Naive assumption $x_{i}$ and $x_{i+1}$ are independent given y
    \[
    \text { i.e. }  p\left(x_{2} \mid x_{1}, y\right)=p\left(x_{2} \mid y\right)
    \]

\end{frame}

\begin{frame}$$
        P(y=1\vert w_{1}=0,w_{2} = 0,w_{3}=1) 
$$
$$ = \frac{P(w_{1}=0|y=1) P(w_{2}=0|y=1) P(w_{3}=1|y=1) P(y=1)}{P(w_{1}=0, w_{2}=0, w_{3}=1)}
    $$
    $$ = \frac{0.6\times 0.8 \times 0.6 \times 0.5}{Z}
    $$
    
\pause Similarly, we can calculate $P(y=0\vert w_{1}=0,w_{2} = 0,w_{3}=1) = \frac{0.6*0.4*0.6*0.5}{Z} $

\pause $\frac{P\left(y=1 \mid w_{1}=0, w_{2}=0, w_{3}=1\right)}{P\left(y=0 \mid w_{1}=0, w_{2}=0, w_{3}=1\right)} = 2 > 1$. Thus, classified as a spam example.
    
\end{frame}

\begin{frame}Note: no cross covariance! Remember all features are independent.

    \includegraphics[width=0.9\textwidth]{kde2d.pdf}
\end{frame}

\begin{frame}{Wikipedia Example}

    \begin{center}
    \begin{tabular}{|c|c|c|c|}
    \hline
    Height&Weight&Footsize&Gender\\
    \hline
    \hline
         6 & 180& 12& M \\
         5.92 & 190& 11& M \\
         5.58 & 170& 12& M \\
         5.92 & 165& 10& M \\
         5 & 100& 6& F \\
         5.5 & 100& 6& F \\
         5.42 & 130& 7& F \\
         5.75 & 150& 7& F \\
         \hline
    \end{tabular}
    
    \end{center}
    
\end{frame}

\begin{frame}{Example}
    \begin{center}

    \begin{tabular}{|c|c|c|}
    \hline
     &Male&Female\\
     \hline
     \hline
     Mean (height) & 5.855 & 5.41  \\
     Variance (height) & 3.5 $\times$ $10^{-2}$ & 9.7 $\times$ $10^{-2}$  \\
     Mean (weight) & 176.25 & 132.5  \\
     Variance (weight) & 1.22 $\times$ $10^{2}$ & 5.5 $\times$ $10^{2}$   \\
     Mean (Foot) & 11.25 & 7.5  \\
     Variance (Foot) & 9.7 $\times$ $10^{-1}$ & 1.67  \\
    \hline
    \hline
    \end{tabular}
    \end{center}
\end{frame}

\begin{frame}{Classify the Person}
    \begin{itemize}
\item Given height = 6ft, weight = 130 lbs, feet = 8 units, classify if it's male or female.
        \item $P(F|6ft, 130 lbs, 8 units) = \dfrac{P(6 ft|F)P(130 lbs|F)P(8 units|F)P(F)}{P(130 lbs, 8 units, 6 ft)}$
        \pause
\item $P(130 lbs|F) = \frac{1}{\sqrt{2\pi\times 550}}\times \exp{\frac{-(132.5-130)^2}{2\times 550}} = .0167$
        \item Finally, we get probability of female given data is greater than the probability of class being male given data.

    \end{itemize}
    
\end{frame}
\end{document}
