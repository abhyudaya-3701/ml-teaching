\documentclass{article}
\usepackage[utf8]{inputenc}
\usepackage{amsmath, amssymb, amsfonts}
\usepackage{bm}
\usepackage{graphicx}
\usepackage{xcolor}
\usepackage{tcolorbox}
\usepackage[bookmarks=false]{hyperref}
\usepackage{url}
\usepackage{booktabs}
\usepackage{array}
\usepackage{multirow}
\usepackage{enumitem}
\usepackage{tikz}
\usetikzlibrary{arrows.meta, positioning, shapes.geometric}
\usepackage[margin=1in]{geometry}
\usepackage{fancyhdr}
\usepackage{titlesec}

\pagestyle{fancy}
\fancyhf{}
\rhead{Tutorial: Linear Regression}
\lhead{ES335 - Machine Learning}
\rfoot{Page \thepage}

\titleformat{\section}{\Large\bfseries\color{blue!75!black}}{\thesection}{1em}{}
\titleformat{\subsection}{\large\bfseries\color{blue!60!black}}{\thesubsection}{1em}{}

\usepackage{../../shared/styles/conventions}

\newcounter{exercise}
\setcounter{exercise}{0}

\title{\textbf{Tutorial: Linear Regression} \\ \textit{Cheat Sheet and Practice Problems}}
\author{ES335 - Machine Learning \\ IIT Gandhinagar}
\date{\today}

\begin{document}

\maketitle

\section{Summary from Slides}

\subsection{Key Concepts}

\textbf{What is Linear Regression?}
\begin{itemize}
    \item Prediction of continuous output variables
    \item Models linear relationship between features and target
    \item Examples: $F = ma$, $v = u + at$
    \item Finds best fit line/hyperplane through data
\end{itemize}

\textbf{Mathematical Model}:
For single feature: 
$$y = \theta_0 + \theta_1 x$$

For multiple features: 
$$y = \theta_0 + \theta_1 x_1 + \theta_2 x_2 + \ldots + \theta_M x_M$$

\subsection{Matrix Formulation}

\textbf{Normal Equation Form}:
$$\hat{\vy} = \mX\vtheta$$

Where:
\begin{itemize}
    \item $\hat{\vy}$: $N \times 1$ predicted output vector
    \item $\mX$: $N \times (M+1)$ design matrix (includes bias column of 1s)
    \item $\vtheta$: $(M+1) \times 1$ parameter vector
    \item $N$: number of samples, $M$: number of features
\end{itemize}

\textbf{Design Matrix Structure}:
$$\mX = \begin{bmatrix}
1 & x_{1,1} & x_{1,2} & \dots & x_{1,M}\\
1 & x_{2,1} & x_{2,2} & \dots & x_{2,M}\\
\vdots & \vdots & \vdots & \dots & \vdots\\
1 & x_{N,1} & x_{N,2} & \dots & x_{N,M}
\end{bmatrix}$$

\subsection{Normal Equation Solution}

\textbf{Objective}: Minimize sum of squared errors
$$\text{MSE} = \frac{1}{N} \sum_{i=1}^N (y_i - \hat{y}_i)^2$$

\textbf{Closed-form Solution}:
$$\vtheta = (\mX\tp\mX)^{-1} \mX\tp\vy$$

\subsection{Basis Expansion}

\textbf{Polynomial Features}: Transform $x$ to $[1, x, x^2, x^3, \ldots]$ to capture non-linear relationships

\textbf{Still Linear}: Model remains linear in parameters $\vtheta$

\subsection{Geometric Interpretation}

\textbf{Projection}: Linear regression projects target vector $\vy$ onto column space of design matrix $\mX$

\textbf{Residuals}: Difference between actual and predicted values, orthogonal to column space

\subsection{Dummy Variables and Multicollinearity}

\textbf{Categorical Variables}: Use one-hot encoding with dummy variables

\textbf{Multicollinearity}: When features are highly correlated
\begin{itemize}
    \item Problem: $\mX\tp \mX$ becomes singular (non-invertible)
    \item Solution: Remove redundant features or use regularization
\end{itemize}

\section{Practice Problems}

\begin{tcolorbox}[colback=gray!5!white,colframe=gray!75!black,title=Problem \stepcounter{exercise}: Basic Linear Regression]

Given data points: $(1,3)$, $(2,5)$, $(3,7)$, $(4,9)$

Find the linear regression line $y = \theta_0 + \theta_1 x$ using normal equation.
\end{tcolorbox}

\begin{tcolorbox}[colback=gray!5!white,colframe=gray!75!black,title=Problem \stepcounter{exercise}: Matrix Setup]

For a dataset with 5 samples and 2 features, write out the complete matrix equation $\hat{\vy} = \mX\vtheta$ with proper dimensions.
\end{tcolorbox}

\begin{tcolorbox}[colback=gray!5!white,colframe=gray!75!black,title=Problem \stepcounter{exercise}: Normal Equation Calculation]

Given design matrix:
$$\mX = \begin{bmatrix} 1 & 2 \\ 1 & 3 \\ 1 & 4 \end{bmatrix}, \quad \vy = \begin{bmatrix} 5 \\ 7 \\ 9 \end{bmatrix}$$

Calculate $\vtheta = (\mX\tp \mX)^{-1} \mX\tp \vy$ step by step.
\end{tcolorbox}

\begin{tcolorbox}[colback=gray!5!white,colframe=gray!75!black,title=Problem \stepcounter{exercise}: MSE Calculation]

For the predictions $\hat{y} = [2.1, 3.9, 6.2, 7.8]$ and actual values $y = [2, 4, 6, 8]$:

a) Calculate the Mean Squared Error
b) Calculate the residuals
c) Verify that residuals sum to zero (approximately)
\end{tcolorbox}

\begin{tcolorbox}[colback=gray!5!white,colframe=gray!75!black,title=Problem \stepcounter{exercise}: Polynomial Regression]

Transform the dataset $x = [1, 2, 3]$ into polynomial features up to degree 3. Write the resulting design matrix.
\end{tcolorbox}

\begin{tcolorbox}[colback=gray!5!white,colframe=gray!75!black,title=Problem \stepcounter{exercise}: Multicollinearity Detection]

Given correlation matrix between three features:
$$R = \begin{bmatrix} 1.0 & 0.9 & 0.1 \\ 0.9 & 1.0 & 0.2 \\ 0.1 & 0.2 & 1.0 \end{bmatrix}$$

Identify which features are highly correlated and explain the potential problems.
\end{tcolorbox}

\begin{tcolorbox}[colback=gray!5!white,colframe=gray!75!black,title=Problem \stepcounter{exercise}: Dummy Variables]

Convert the categorical variable ``Color'' with values [Red, Blue, Green, Red, Blue] into dummy variables. Show the resulting design matrix.
\end{tcolorbox}

\begin{tcolorbox}[colback=gray!5!white,colframe=gray!75!black,title=Problem \stepcounter{exercise}: Geometric Interpretation]

Explain why the residual vector is orthogonal to the column space of the design matrix. What does this mean geometrically?
\end{tcolorbox}

\begin{tcolorbox}[colback=gray!5!white,colframe=gray!75!black,title=Problem \stepcounter{exercise}: Non-invertible Matrix]

When does $(\mX\tp \mX)$ become non-invertible? Give three specific scenarios and explain how to handle each case.
\end{tcolorbox}

\begin{tcolorbox}[colback=gray!5!white,colframe=gray!75!black,title=Problem \stepcounter{exercise}: Feature Scaling]

Dataset before scaling: $\mX = \begin{bmatrix} 1000 & 2 \\ 2000 & 4 \\ 3000 & 6 \end{bmatrix}$

Apply standardization (z-score normalization) to both features. Explain why scaling might be important for linear regression.
\end{tcolorbox}

\begin{tcolorbox}[colback=gray!5!white,colframe=gray!75!black,title=Problem \stepcounter{exercise}: Model Selection]

You have three models:
- Model 1: $y = \theta_0 + \theta_1 x$ (MSE = 4.2)
- Model 2: $y = \theta_0 + \theta_1 x + \theta_2 x^2$ (MSE = 3.8)  
- Model 3: $y = \theta_0 + \theta_1 x + \theta_2 x^2 + \theta_3 x^3$ (MSE = 3.9)

Which model would you choose and why? Consider overfitting.
\end{tcolorbox}

\begin{tcolorbox}[colback=gray!5!white,colframe=gray!75!black,title=Problem \stepcounter{exercise}: Residual Analysis]

After fitting a linear regression model, you observe that residuals show a curved pattern when plotted against fitted values. What does this indicate and how would you address it?
\end{tcolorbox}

\begin{tcolorbox}[colback=gray!5!white,colframe=gray!75!black,title=Problem \stepcounter{exercise}: Computational Complexity]

Compare the computational complexity of:
a) Normal equation method: $(\mX\tp \mX)^{-1} \mX\tp \vy$
b) Gradient descent method

When would you prefer each approach?
\end{tcolorbox}

\begin{tcolorbox}[colback=gray!5!white,colframe=gray!75!black,title=Problem \stepcounter{exercise}: Real-world Application]

Design a linear regression model to predict house prices with features:
- Size (sq ft)
- Number of bedrooms  
- Age of house
- Distance to city center

a) Write the mathematical model
b) Identify potential multicollinearity issues
c) Suggest preprocessing steps
d) How would you validate the model?
\end{tcolorbox}

\begin{tcolorbox}[colback=gray!5!white,colframe=gray!75!black,title=Problem \stepcounter{exercise}: Advanced Challenge]

Given that normal equation requires matrix inversion which is $O(n^3)$:

a) For what size datasets does this become computationally prohibitive?
b) Explain why iterative methods like gradient descent might be preferred
c) What are the trade-offs between exact solution (normal equation) vs approximate solution (gradient descent)?
\end{tcolorbox}

\end{document}